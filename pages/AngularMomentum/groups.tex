\subsection{SO(3) and SU(2) group}

一些我们可能会用到的群:

\begin{itemize}
  \item ${\rm GL}(n, \mathbb{F})$, General Linear Group, 数域$\mathbb{F}$ 上任意有逆的$n\times n$ 矩阵构成的群.
  \item ${\rm O}(n)$, Orthogonal Group, 对应于$n\times n$ 正交矩阵, 满足$\forall Q\in {\rm O}(n), \det Q = \pm 1$.
  \item ${\rm SO}(n)$, Sepcial Orthogonal Group, ${\rm O}(n)$ 群中行列式为 $+1$ 的元素构成的子群.
  \item ${\rm U}(n)$, Unitary Group, $n\times n$ 的幺正矩阵.
  \item ${\rm SU}(n)$, Special Unitary Group, ${\rm U}(n)$ 群中行列式为$+1$ 的元素构成的群.
\end{itemize}

\begin{remark}
  讨论一下${\rm SO}(n)$ 群特殊/正当在哪. 首先不难看出, 由于${\rm O}(n)$ 的恒等元 $E$ 满足
  $\det E = 1$,  并且对于$\forall Q_1, Q_2\in {\rm SO}(n), \det Q_1 = \det Q_2 = -1$,
  有$\det(Q_1 Q_2) = +1$, 因此${\rm O}(n)$中满足$\det Q=-1$ 的元素无法构成子群. 另外
  我们一般将${\rm SO}(3)$ 群看作是转动群, 注意到空间反演算符$P \vec{r} \to -\vec{r}$,
  可以看出$P = -I, \det P = 1$, 因此我们可以利用空间反演算符在正当转动和非正当转动
  中构建同构: \emph{在${\rm O}(3)$ 群中, 非正当转动等于正当转动后再进行一次空间反演,反之亦然}.
  这也解释了为什么我们不用${\rm O}(3)$ 群而使用${\rm SO}(3)$ 群: 空间反演和空间转动是不同的对称性.
\end{remark}

而在前面, 通过对Spin $\frac{1}{2}$ 系统的讨论, 我们看到了可以将旋转算符写成一个
$2\times 2$ 的复矩阵的形式, 并且注意到式(3.2.45), 这是一个幺正矩阵, 进一步还满足
行列式等于$+1$. 这给予我们一个启发: 转动也可以通过${\rm SU}(2)$ 群来表示.

那么很自然的, 我们希望比较${\rm SO}(3)$ 和${\rm SU}(2)$ 的关系, 我们早已知道结论:
\emph{存在一个从${\rm SU}(2)$ 到${\rm SO}(3)$ 的$2$ 对$1$ 的同态}. Sakurai和喀兴林上
给出的计算证明已经足够充足, 这里我们给出一个Lie群流形上的理解:

\begin{remark}
  不难看出的是, 对于${\rm SU}(2)$ 群, 我们可以看出其的Lie流形是$S_3$.
  这是因为一个${\rm SU}(2)$ 的群元一定可以写为:

  \begin{equation}
    \begin{aligned}
      u =
      \begin{pmatrix}
        x_1 + {\rm i} x_2 & x_3 + {\rm i} x_4 \\
        -x_3 + {\rm i} x_4 & x_1 - {\rm i} x_2
      \end{pmatrix}
    \end{aligned}
  \end{equation}

  而$\det u=1$ 给出:
  
  \begin{equation}
    \begin{aligned}
      x_1^{2} + x_2^{2} + x_3^{2} + x_4^{2} = 1
    \end{aligned}
  \end{equation}

  这正是$S_3$ 球面. 而对于${\rm SO}(3)$, 由于可以用三维转动角来进行参数化, 因此
  可以很自然的和$S_3$ 球面联系起来. 但是注意到, ${\rm SO}(3)$ 流形上需要对认同,
  这是由于顺时针转$\pi$ 角和逆时针转$\pi$ 角完全相同. 因此${\rm SO}(3)$ 的Lie流形
  实际上是射影空间${\rm \mathbb{R}P}_3$. 因此, 不难看出${\rm SO}(3)$ 是${\rm SU}(2)$ 
  将对径点视作等价类得到的商空间.

\end{remark}

现在我们来研究${\rm SO}(3)$ 的一些性质:

\begin{lemma}[Tensor Invariant 1]
  对于任意$R\in {\rm SO}(3)$, 有:
  \mn{这里以及下面默认使用了Einstein求和约定}
  \begin{equation}\label{eq:SO3_tensor_invariant_1}
    \begin{aligned}
      R^{i}_{k}R^{j}_{l}\delta^{kl} = \delta^{ij}
    \end{aligned}
  \end{equation}
  
\end{lemma}
\begin{proof}
  \begin{equation}
    \begin{aligned}
      R^{i}_{k}R^{j}_{l}\delta^{kl} &= R^{i}_{k}R^{j}_{l} = R^{i}_{k}(R^{T})^{k}_{j} \\
        &= (RR^{T})^{i}_{j} = \delta_{ij}
    \end{aligned}
  \end{equation}
\end{proof}

\begin{lemma}[Tensor Invariant 2]
  对于任意$R\in {\rm SO}(3)$, 有:
  \begin{equation}\label{eq:SO3_tensor_invariant_2}
    \begin{aligned}
      R^{i}_{l}R^{j}_{m}R^{k}_{n} \epsilon^{lmn} = \epsilon^{ijk}
    \end{aligned}
  \end{equation}

\end{lemma}
\begin{proof}
  首先注意到矩阵的行列式可以写成:

  \begin{equation}
    \begin{aligned}
      \det R = R^{1}_{i}R^{2}_{k}R^{3}_{l} \epsilon^{lmn}
    \end{aligned}
  \end{equation}

  因此, 对于式\ref{eq:SO3_tensor_invariant_2},
  当$i,j,k$ 存在重复时, 对应于存在相同行
  的矩阵的行列式, 因此${\rm LHS}=0$; 而当$i,j,k$ 不
  重复时,  由于行列式每交换一次行结果就会变为相反数,
  结合Levi-Civita符号的逆序数定义即可证明.
\end{proof}

不妨回过头来考虑一下定轴转动的情况: 以$x,y,z$ 
三个轴做定轴转动, 记作$R_1, R_2, R_3$,
不难看出$R_1,R_2,R_3$ 均构成${\rm SO}(2)$ 
群, 并有如下表示:

\begin{equation}\label{eq:r1_mat}
  \begin{aligned}
    R_1(\psi) =
    \begin{pmatrix}
      1 & 0 & 0 \\
      0 & \cos \psi & -\sin \psi \\
      0 & \sin \psi & \cos \psi
    \end{pmatrix}
  \end{aligned}
\end{equation}

\begin{equation}\label{eq:r2_mat}
  \begin{aligned}
    R_{2}(\psi) =
    \begin{pmatrix}
      \cos \psi & 0 & \sin \psi \\
      0 & 1 & 0 \\
      -\sin \psi & 0 & \cos \psi
    \end{pmatrix}
  \end{aligned}
\end{equation}

\begin{equation}\label{eq:r3_mat}
  \begin{aligned}
    R_{3}(\psi) =
    \begin{pmatrix}
      \cos \psi & -\sin \psi & 0 \\
      \sin \psi & \cos \psi & 0\\
      0 & 0 & 1
    \end{pmatrix}
  \end{aligned}
\end{equation}

同时, 考虑绕矢量$\vec{n}$ 的转动$R_{n}$ 的
生成元$J_{n}$:

\begin{equation}\label{eq:J_generator_infinitesimal}
  \begin{aligned}
    \lim_{\psi\to 0} R_{n}(\psi) = 1 - {\rm i} \epsilon J_{n}
  \end{aligned}
\end{equation}

或者:

\begin{equation}\label{eq:J_generator_exponential}
  \begin{aligned}
    R_{n}(\psi) = e^{- {\rm i} \psi J_{n}}
  \end{aligned}
\end{equation}

\begin{lemma}[ $J_{n}$ 的矢量性]
  给定$\vec{n}$ 以及一个三维转动 $R$, 使得
  $R \vec{n}R^{-1}=\vec{n'}$:

  \begin{equation}
    \begin{aligned}
      R J_{n} R^{-1} = J_{n'}
    \end{aligned}
  \end{equation}

\end{lemma}
\begin{proof}
  利用矩阵恒等式:
  \begin{equation}
    \begin{aligned}
      R e^{- {\rm i}\psi J_{n}}R^{-1} = e^{- {\rm i}\psi R J_{n} R^{-1}}
    \end{aligned}
  \end{equation}
\end{proof}

我们再利用$J_{n}$ 的无穷小形式\ref{eq:J_generator_infinitesimal},
从式\ref{eq:r1_mat}-式\ref{eq:r3_mat}, 可以写
出$J_1, J_{2}, J_3$ 的表示:

\begin{equation}
  \begin{aligned}
    J_1 =
    \begin{pmatrix}
      0 & 0 & 0 \\
      0 & 0 & - {\rm i} \\
      0 & {\rm i} & 0 \\
    \end{pmatrix}\quad
    J_2 =
    \begin{pmatrix}
      0 & 0 & {\rm i}\\
      0 & 0 & 0\\
      - {\rm i} & 0 & 0
    \end{pmatrix}\quad
    J_3 = 
    \begin{pmatrix}
      0 & - {\rm i} & 0\\
      {\rm i} & 0 & 0\\
      0 & 0 & 0
    \end{pmatrix}
  \end{aligned}
\end{equation}

利用Levi-Civita记号, 可以写成:

\begin{equation}\label{eq:lc_form_j_n}
  \begin{aligned}
    (J_{k})^{l}_{m} = - {\rm i} \epsilon_{klm}
  \end{aligned}
\end{equation}

\begin{theorem}[Vector Generator $J$]
  \begin{enumerate}
    \item 对于$J_{i}, i\in \{1,2,3\}$, 存在关系:
      \begin{equation}
        \begin{aligned}
          R J_{k} R^{-1} = J_{l} R^{l}_{k}
        \end{aligned}
      \end{equation}

    \item 对于绕任意轴$\vec{n}$ 的生成元$J_{n}$, 存在分解:
      \begin{equation}\label{eq:j_decomposition}
        \begin{aligned}
          J_{n} = J_{k} n_{k},\quad \text{where } \vec{n} = n_{k}\vec{e}_{k}
        \end{aligned}
      \end{equation}
  \end{enumerate}
\end{theorem}
\begin{proof}
  \begin{enumerate}
    \item 利用引理\ref{eq:SO3_tensor_invariant_2}, 两边同乘$R^{i}_{s}$ 并对$i$ 求和:
      \begin{equation}
        \begin{aligned}
          R^{i}_{s}R^{i}_{l}R^{j}_{m}R^{k}_{n}\epsilon^{lmn} = R^{i}_{s}\epsilon^{ijk}
        \end{aligned}
      \end{equation}
      注意到$R^{i}_{s}R^{i}_{l}$ 实际上就是$(R^{T}R)^{s}_{l}=\delta^{sl}$,
      因此有:
      \begin{equation}
        \begin{aligned}
          R^{j}_{m}R^{k}_{n}\epsilon^{smn} = R^{i}_{s}\epsilon^{ijk}
        \end{aligned}
      \end{equation}
      同时利用式\ref{eq:lc_form_j_n}, 可以得到:
      \begin{equation}
        \begin{aligned}
          R^{j}_{m}R^{k}_{n}(J_{s})^{m}_{n} = R^{i}_{s}(J_{i})^{j}_{k}
        \end{aligned}
      \end{equation}
      至此证毕.
    \item 利用1中结论易证.
  \end{enumerate}
\end{proof}

式\ref{eq:j_decomposition}指出$J_1, J_2, J_3$ 构成
所有${\rm SO}(3)$ 的单参阿贝尔子群的生成元的一组基.

\begin{theorem}[Lie algebra of ${\rm SO}(3)$]
  \begin{equation}\label{eq:SO3_lie_algebra}
    \begin{aligned}
      [J_{j}, J_{k}] = {\rm i} \epsilon_{jkl}J^{l}
    \end{aligned}
  \end{equation}
\end{theorem}
\begin{proof}
  首先, 当$j=k$ 时, 上式毫无疑问的成立. 当$j\neq k$ 时,
  我们以$j=1, k=2$ 为例, 考虑对$J_1$ 做一个绕$\vec{e}_{2}$ 的
  无穷小转动$R_2(\dd \psi) = I - {\rm i} \dd \psi J_2$:
  \begin{equation}
    \begin{aligned}
      R_2(\dd \psi)J_1 R_2^{-1}(\dd \psi) = J_{k}(R_2(\dd \psi))^{k}_{1}
    \end{aligned}
  \end{equation}
  仅保留$\dd \psi$的一次项:
   \begin{equation}
    \begin{aligned}
      {\rm LHS} &= (I - {\rm i}\dd \psi J_2)J_1 (I+{\rm i}\dd \psi J_2)\\
                &= J_1 + {\rm i}\dd \psi[J_1, J_2]
    \end{aligned}
  \end{equation}
  同理, 对于右侧:
  \begin{equation}
    \begin{aligned}
      {\rm RHS} &= J_{k}(I- {\rm i}\dd \psi J_2)^{k}_{1} \\
                &= J_1 - {\rm i}\dd \psi J_{k} (J_2)^{k}_{1}
    \end{aligned}
  \end{equation}
  代入$J_2$ 的具体形式, 有:
  \begin{equation}
    \begin{aligned}
      {\rm RHS} = J_1 -\dd \psi J_{3}
    \end{aligned}
  \end{equation}
  对比可知:
  \begin{equation}
    \begin{aligned}
      [J_1, J_2] = {\rm i} J_{3}
    \end{aligned}
  \end{equation}
  类似的, 可以证明其他情况.
\end{proof}

\begin{remark}
  式\ref{eq:SO3_lie_algebra}在恒等元附近给出与群乘法
  一样的信息, 但是还存在一些"全局"性质, 例如 $R_{n}(2 \pi)=E$,
  $R_{n}(\pi)=R_{-n}(\pi)$等, 这些性质为群表示添加了额外的限制.
\end{remark}
