\section{Angular Momentum}

\subsection{SO(3) and SU(2) group}

一些我们可能会用到的群:

\begin{itemize}
  \item ${\rm GL}(n, \mathbb{F})$, General Linear Group, 数域$\mathbb{F}$ 上任意有逆的$n\times n$ 矩阵构成的群.
  \item ${\rm O}(n)$, Orthogonal Group, 对应于$n\times n$ 正交矩阵, 满足$\forall Q\in {\rm O}(n), \det Q = \pm 1$.
  \item ${\rm SO}(n)$, Sepcial Orthogonal Group, ${\rm O}(n)$ 群中行列式为 $+1$ 的元素构成的子群.
  \item ${\rm U}(n)$, Unitary Group, $n\times n$ 的幺正矩阵.
  \item ${\rm SU}(n)$, Special Unitary Group, ${\rm U}(n)$ 群中行列式为$+1$ 的元素构成的群.
\end{itemize}

\begin{remark}
  讨论一下${\rm SO}(n)$ 群特殊/正当在哪. 首先不难看出, 由于${\rm O}(n)$ 的恒等元 $E$ 满足
  $\det E = 1$,  并且对于$\forall Q_1, Q_2\in {\rm SO}(n), \det Q_1 = \det Q_2 = -1$,
  有$\det(Q_1 Q_2) = +1$, 因此${\rm O}(n)$中满足$\det Q=-1$ 的元素无法构成子群. 另外
  我们一般将${\rm SO}(3)$ 群看作是转动群, 注意到空间反演算符$P \vec{r} \to -\vec{r}$,
  可以看出$P = -I, \det P = 1$, 因此我们可以利用空间反演算符在正当转动和非正当转动
  中构建同构: \emph{在${\rm O}(3)$ 群中, 非正当转动等于正当转动后再进行一次空间反演,反之亦然}.
  这也解释了为什么我们不用${\rm O}(3)$ 群而使用${\rm SO}(3)$ 群: 空间反演和空间转动是不同的对称性.
\end{remark}

而在前面, 通过对Spin $\frac{1}{2}$ 系统的讨论, 我们看到了可以将旋转算符写成一个
$2\times 2$ 的复矩阵的形式, 并且注意到式(3.2.45), 这是一个幺正矩阵, 进一步还满足
行列式等于$+1$. 这给予我们一个启发: 转动也可以通过${\rm SU}(2)$ 群来表示.

那么很自然的, 我们希望比较${\rm SO}(3)$ 和${\rm SU}(2)$ 的关系, 我们早已知道结论:
\emph{存在一个从${\rm SU}(2)$ 到${\rm SO}(3)$ 的$2$ 对$1$ 的同态}. Sakurai和喀兴林上
给出的计算证明已经足够充足, 这里我们给出一个Lie群流形上的理解:

\begin{remark}
  不难看出的是, 对于${\rm SU}(2)$ 群, 我们可以看出其的Lie流形是$S_3$.
  这是因为一个${\rm SU}(2)$ 的群元一定可以写为:

  \begin{equation}
    \begin{aligned}
      u =
      \begin{pmatrix}
        x_1 + {\rm i} x_2 & x_3 + {\rm i} x_4 \\
        -x_3 + {\rm i} x_4 & x_1 - {\rm i} x_2
      \end{pmatrix}
    \end{aligned}
  \end{equation}

  而$\det u=1$ 给出:
  
  \begin{equation}
    \begin{aligned}
      x_1^{2} + x_2^{2} + x_3^{2} + x_4^{2} = 1
    \end{aligned}
  \end{equation}

  这正是$S_3$ 球面. 而对于${\rm SO}(3)$, 由于可以用三维转动角来进行参数化, 因此
  可以很自然的和$S_3$ 球面联系起来. 但是注意到, ${\rm SO}(3)$ 流形上需要对认同,
  这是由于顺时针转$\pi$ 角和逆时针转$\pi$ 角完全相同. 因此${\rm SO}(3)$ 的Lie流形
  实际上是射影空间${\rm \mathbb{R}P}_3$. 因此, 不难看出${\rm SO}(3)$ 是${\rm SU}(2)$ 
  将对径点视作等价类得到的商空间.

\end{remark}

\subsection{纯态, 混合态与密度算符}

迄今为止, 我们讨论的态都是Hilbert空间中的态, 这样的态我们称之为 {\bf 纯态}, 但一些
更复杂的情况下, 物理系统的态可能分别以$p_1, p_2, \cdots, p_{n}$的概率处于态$\ket{a_1}$, $\ket{a_2}$,$\cdots$, $\ket{a_{n}}$,
这样的情况我们称之为{\bf 混合态}, 记作:

\begin{equation}
  \left\{
    \begin{aligned}
      &\ket{a_1}: p_1 \\
      &\ket{a_2}: p_2 \\
      &\cdots \\
      &\ket{a_n}: p_{n}
    \end{aligned}
  \right.
\end{equation}

\begin{remark}
  请注意区分纯态$c_1\ket{a_1} + c_2\ket{a_2}+\cdots+c_{n}\ket{a_{n}}$与上面的混合态,
  即使$|c_{i}|^{2} = p_{i}$. 这一点可以通过讨论另一个力学量$B$ 的期望看出.
  假设$B$ 的本征矢量为$\ket{b_i}$, 那么对于纯态$\ket{\alpha}$,  $B$ 取$b_{i}$ 的概率为:
  
  \begin{equation}
    \begin{aligned}
      |\bra{b_{i}}\ket{\alpha}|^{2} = |\bra{b_{i}}\ket{a_1}c_1 + \bra{b_{i}}\ket{a_2}c_2 + \cdots + \bra{b_{i}}\ket{a_{n}}c_{n}|^{2}
    \end{aligned}
  \end{equation}

  而对于混合态$\ket{\beta}$, B取$b_{i}$的概率为:

  \begin{equation}
    \begin{aligned}
      \sum_{j}^{n} p_{j} |\bra{b_{i}}\ket{a_{j}}|^{2}
    \end{aligned}
  \end{equation}

  可以看出, 对于纯态, 发生的是概率幅之间的\emph{相干叠加}, 而混合态则是概率的\emph{不相干叠加}
\end{remark}

我们希望找到一种既能描述混合态也能描述纯态的数学量, 这使得我们引入了密度算符:

\begin{equation}
  \begin{aligned}
    \rho = \sum_{i} p_{i} \ket{a_{i}}\bra{a_{i}}
  \end{aligned}
\end{equation}

不难看出, 对于混合态, 任意力学量$A$ 的期望值为:

\begin{equation}
  \begin{aligned}
    \langle A\rangle = \sum_{i} \bra{i}A \rho \ket{i} = {\rm tr} \;A \rho
  \end{aligned}
\end{equation}

由于迹与表象无关, 因此上式中的基矢$\ket{i}$ 可以任意选取.

\begin{theorem}
  密度算符 $\rho$ 是个厄密算符.
\end{theorem}
\begin{proof}
  给定任意一组基矢$\ket{n}$, 将密度算符$\rho$ 在此基矢下展开, 可以发现:
  \begin{equation}
    \begin{aligned}
      \rho_{mn} = \sum_{i}p_{i}\bra{m}\ket{a_{i}}\bra{a_{i}}\ket{n} = \rho_{nm}^{*}
    \end{aligned}
  \end{equation}
  即$\rho$ 是个厄密矩阵.
\end{proof}

\begin{theorem}
  对于纯态的密度算符$\rho$, 满足:
  \begin{equation}
    \begin{aligned}
      \rho^{2} = \rho
    \end{aligned}
  \end{equation}
\end{theorem}
\begin{proof}
  易证, 略
\end{proof}

\begin{theorem}
  对于密度算符$\rho$ 的迹, 我们有:
  \begin{equation}
    \begin{aligned}
      {\rm tr} \; \rho = 1
    \end{aligned}
  \end{equation}

  \begin{equation}
    {\rm tr}\; \rho^{2} \left\{
    \begin{aligned}
      & = 1, \text{纯态}\\
      & < 1, \text{混合态}
    \end{aligned}\right.
  \end{equation}
\end{theorem}
\begin{proof}
  \begin{equation}
    \begin{aligned}
      {\rm tr}\; \rho &= \sum_{n} \sum_{i} p_{i}\bra{n}\ket{a_{i}}\bra{a_{i}}\ket{n} \\
                      &= \sum_{i} \sum_{n} p_{i} \bra{a_{i}}\ket{n}\bra{n}\ket{a_{i}} \\
                      &= \sum_{i} p_{i} = 1
    \end{aligned}
  \end{equation}

  \begin{equation}
    \begin{aligned}
      {\rm tr}\; \rho^{2} &= \sum_{n} \sum_{ij} p_{i} p_{j}\bra{n}\ket{a_{i}}\bra{a_{i}}\ket{a_{j}}\bra{a_{j}}\ket{n}\\
                          &= \sum_{ij}p_{i}p_{j} \bra{a_{j}}\ket{a_{i}}\bra{a_{i}}\ket{a_{j}}\\
                          &= \sum_{i} p_{i} \left(\sum_{j} p_{j} |\bra{a_{i}}\ket{a_{j}}|^{2}\right)
    \end{aligned}
  \end{equation}
  上式中, 如果为纯态, 那么退化为 ${\rm tr}\; \rho^{2} = 1$, 如果为混合态, 当$i\neq j$
  时$|\bra{a_{i}}\ket{a_{j}}|^{2}<1$, 则有
  \begin{equation}
    \begin{aligned}
      {\rm tr}\; \rho^{2} < \sum_{i}p_{i} = 1
    \end{aligned}
  \end{equation}
\end{proof}

\begin{theorem}
  若混合态由一系列相互正交的态构成, 则密度算符$\rho$ 的本征态即为那些构成混合态的态$\ket{a_{i}}$,
  相应的本征值为概率$p_{i}$.
\end{theorem}
\begin{proof}
  易证, 略
\end{proof}
\begin{remark}
  对于那些混合态由不相互正交的态构成的情况, 由于密度算符仍然是厄密算符, 它一定具有一系列相互正交的本征态$\ket{b_{i}}$:
  \begin{equation}
    \begin{aligned}
      \rho \ket{b_{i}} = q_{i} \ket{b_{i}}
    \end{aligned}
  \end{equation}
  那么将$\rho$ 在$\ket{b_{i}}$ 基矢下展开为对角阵形式:
  \begin{equation}
    \begin{aligned}
      \rho = \sum_{i} q_{i} \ket{b_{i}}\bra{b_{i}}
    \end{aligned}
  \end{equation}
  可以看出, 我们可以用另一组相互正交态构成一个相同的密度算符. 那么这两个混合态是
  相同的吗?
\end{remark}


在Heisenberg绘景下, 态矢量$\ket{a_{i}}^{H}$ 不含时, 因此密度算符也是个不含时算符:
\begin{equation}
  \begin{aligned}
    \rho^{H} = \sum_{i} p_{i} \ket{a_{i}}^{H}\bra{a_{i}}^{H}
  \end{aligned}
\end{equation}

在Schr\"odinger绘景下, 密度算符是一个含时算符, 其运动方程可以由Schr\"odinger方程导出:
\begin{equation}\label{eq:density_op_Liouville_eq}
  \begin{aligned}
    - {\rm i} \hbar \pdv{\rho^{S}(t)}{t} = [H, \rho^{S}(t)]
  \end{aligned}
\end{equation}

\subsection{复合体系与约化密度矩阵}
假设现在有这样的情况, 我们考虑一个有两个粒子$1,2$ 构成的复合体系, 但我们现在
只想求粒子$1$ 的某个物理量$A_1$, 会得到怎样的结果呢?

对于双粒子体系, 复合体系的基矢定义在两粒子Hilbert空间$\mathcal{H}_{1}, \mathcal{H}_{2}$的张量积空间上
$\mathcal{H}_{1}\otimes \mathcal{H}_{2}$, 任意纯态的态矢$\ket{\alpha}$为:
\begin{equation}
  \begin{aligned}
    \ket{\alpha} = \sum_{i} \sum_{m} \ket{i,m} c_{im}
  \end{aligned}
\end{equation}
其中:
\begin{equation}
  \begin{aligned}
    \ket{i,m} = \ket{a_{i}}\otimes \ket{b_{m}}
  \end{aligned}
\end{equation}

同时满足归一化条件:
\begin{equation}
  \begin{aligned}
    \sum_{i} \sum_{m} |c_{im}|^{2} = 1
  \end{aligned}
\end{equation}
为了简单, 我们考虑复合系统处于纯态时的密度算符:
\begin{equation}
  \begin{aligned}
    \rho = \ket{\alpha}\bra{\alpha} = \sum_{ii'}\sum_{mm'}c_{i'm'}c_{im}^{*}\ket{i', m'}\bra{i, m}
  \end{aligned}
\end{equation}
现在我们考虑粒子$1$ 的力学量$A_1$ 的期望值:
\begin{equation}
  \begin{aligned}
    \langle A_1 \rangle &= {\rm tr}\; (A_1 \rho) \\
                        &= \sum_{jj'}\sum_{nn'}\bra{j', n'}A_1\ket{j, n}\bra{j, n}\rho \ket{j', n'}\\
                        &= \sum_{jj'}\bra{a_{j'}}A_1\ket{a_{j}}\sum_{n}\bra{j,n}\rho \ket{j', n}
  \end{aligned}
\end{equation}

令$\rho_1$ 为:
\begin{equation}
  \begin{aligned}
    \rho_1 = \sum_{n} \bra{b_{n}}\rho \ket{b_{n}} = {\rm tr}_2 \rho
  \end{aligned}
\end{equation}
其中${\rm tr}_{2}$ 表示求偏迹, 即仅在$\mathcal{H}_{2}$中求迹. 得到的$\rho_1$ 是$\mathcal{H}_{1}$ 中的算符.
称作粒子$1$ 的约化密度算符.

利用约化密度算符$\rho_1$, 我们可以求得$A_1$ 的平均值:
\begin{equation}
  \begin{aligned}
    \langle A_1 \rangle = {\rm tr}_{1}(A_1 \rho_1)
  \end{aligned}
\end{equation}
上式仅在$\mathcal{H}_{1}$ 中求值.

但值得注意的是, 尽管我们讨论的是一个双粒子纯态, 但是约化密度算符 $\rho_1$ 对应的态
可能是一个混合态, 这一现象称为\emph{纠缠}. 进一步不做证明的, 如果一个复合态$\rho_{123\cdots}$ 
可以写为$\rho_1\otimes \rho_2 \otimes \rho_3\otimes \cdots$的形式, 则称其为直积态, 其各约化密度算符
对应于纯态; 反之则称为纠缠态.

\subsection{量子熵}

现在让我们来简单讨论一下量子体系中熵的概念. 首先, 我们希望这一量子熵刻画某种不确定性,
这种不确定性是由于\emph{混合态中各态按照概率随机出现}导致的; 并且为了和热力学熵相似,
我们希望量子熵也是广延量\sn{这里的数学处理会更加复杂, 原因在于每个量子体系拥有自己独立的Hilbert空间,
而混合体系的Hilbert空间应当是各体系空间的张量积.}, 我们先定义一个算符:
\sn{此处的负号是为了使$o$ 半正定}
\begin{equation}
  \begin{aligned}
    o = -\log \rho
  \end{aligned}
\end{equation}

首先不难看出, 这一定义满足广延量的性质. 并且, 对于纯态:

\begin{theorem}
若$\rho$ 是一个描述纯态的密度算符, 则有:  
\begin{equation}
  \begin{aligned}
    o(\rho) = 0
  \end{aligned}
\end{equation}
\end{theorem}
\begin{proof}
  由于纯态的密度算符的幂等性, 有:
  \begin{equation}
    \begin{aligned}
      \log(\rho^{2}) = \log \rho + \log \rho = \log \rho
    \end{aligned}
  \end{equation}
  即$\log \rho=0$
\end{proof}

符合我们的直观(刻画的是由混合态中态的混合引入的不确定性).

那么, 这样的一个算符, 其在相应的密度算符$\rho$ 刻画的态下的期望值为:

\begin{equation}
  \begin{aligned}
    S = -{\rm tr}\; (\rho \log \rho)
  \end{aligned}
\end{equation}

