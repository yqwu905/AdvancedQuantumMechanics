\subsection{Wigenr-Eckart Theorem}

对于给定本征态$\ket{j\,m},m=j,j-1,\cdots,-j$, 对于任意$g\in G$,
总有:

\begin{equation}
  \begin{aligned}
    U(g)\ket{j\,m} = D^{j}_{m'm}(g) \ket{j\,m'}, \quad \forall g\in G
  \end{aligned}
\end{equation}

由于$D^{j}_{m'm}(g)$ 是不可约表示, 由$\ket{j,m}$ 张成的$2j+1$ 
维的空间也是转动下的不变子空间. 而$\ket{j,m}$ 成为表示$j$ 的
不可约基矢.

同样的, 对于算符, 也可以类似的定义:

\begin{equation}
  \begin{aligned}
    U(g)T^{k}_{q}U(g)^{-1} = \sum_{q'}T^{k}_{q'}D^{k}_{q'q}(g), \quad \forall g\in G
  \end{aligned}
\end{equation}

这样的一组算符$T^{k}_{q}$ 称为表示$k$ 的不可约张量算符.

\begin{remark}
  对于我们要讨论的${\rm SO}(3)$ 群, 上述定义可以进一步称为
  不可约球张量算符.
\end{remark}

进一步的, 在${\rm SO}(3)$ 群的条件下, 不难证明:

\begin{equation}
  \begin{aligned}
    &[J_{z}, T^{q}_{k}] = q T^{k}_{q}\\
    &[J_{\pm}, T^{k}_{q}] = \sqrt{k(k+1) - q(q\pm 1)} T^{k}_{q\pm 1}
  \end{aligned}
\end{equation}

不可约张量的重要性质就是本章的重点Wigner-Eckart定理:

\begin{theorem}[Wigner-Eckart Theorem]
  \begin{equation}
    \begin{aligned}
      \bra{\alpha';j'\,m'}T^{k}_{q}\ket{\alpha;j\,m} = \bra{j\,m;k\,q}\ket{j'\,m'}\frac{1}{\sqrt{2j+1}}\bra{\alpha';j'}T^{k}\ket{\alpha;j}
    \end{aligned}
  \end{equation}
\end{theorem}

这一定理的重要性在于可以让我们将磁量子数分离出来, 同时利用CG系数
来选择合法的情况, 同时对于$\alpha=\alpha', j=j'$ 时, 跃迁概率
直接等于CG系数之比.
