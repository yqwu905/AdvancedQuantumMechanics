\subsection{Wigenr-Eckart Theorem}

张量算符在群变换下的作用规律指出, 张量算符的分量是群的一个
表示空间的基矢.

我们可以通过两个矢量算符${\bf V} = \{V_{i}\}, {\bf U} = \{U_{i}\}$ 
的并矢构造出一个张量$T_{ij} = U_{i}V_{j}$. 对于${\rm SO}(3)$群,
这个张量对应于一个9维的表示空间. 但是这个空间是在群作用下可约
的, 这是因为上面的构造出来的张量是一个二阶张量, 因此存在可以分解
成 $1+3+5$ 的迹,反对称以及无迹对称张量. 而群作用总是将反对称
张量变为反对称张量,无迹对称张量变为无迹对称张量... 这意味着
我们可以将9维的表示空间中存在$1,3,5$维的不变子空间, 这意味着
表示是可约的.

为了构造不可约表示, 我们希望张量算符是一阶张量, 在${\rm SO}(3)$ 群
的语境下, 我们将其称作{\bf 不可约球张量}.

进一步的, 在${\rm SO}(3)$ 群的条件下, 不难证明:

\begin{equation}
  \begin{aligned}
    &[J_{z}, T^{q}_{k}] = q T^{k}_{q}\\
    &[J_{\pm}, T^{k}_{q}] = \sqrt{k(k+1) - q(q\pm 1)} T^{k}_{q\pm 1}
  \end{aligned}
\end{equation}

不可约张量的重要性质就是本章的重点Wigner-Eckart定理:

\begin{theorem}[Wigner-Eckart Theorem]
  \begin{equation}
    \begin{aligned}
      \bra{\alpha';j'\,m'}T^{k}_{q}\ket{\alpha;j\,m} = \bra{j\,m;k\,q}\ket{j'\,m'}\frac{1}{\sqrt{2j+1}}\bra{\alpha';j'}T^{k}\ket{\alpha;j}
    \end{aligned}
  \end{equation}
\end{theorem}

这一定理的重要性在于可以让我们将磁量子数分离出来, 同时利用CG系数
来选择合法的情况, 同时对于$\alpha=\alpha', j=j'$ 时, 跃迁概率
直接等于CG系数之比.
