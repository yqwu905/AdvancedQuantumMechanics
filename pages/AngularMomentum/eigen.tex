\subsection{角动量的本征态与本征值\sn{
也不妨叫做${\rm SO}(3)$ 群李代数的不可约表示}}

\begin{theorem}
  定义算符$J^{2} = J_1^{2} + J_2^{2} + J_3^{2}$,有:
  \begin{equation}
    \begin{aligned}
      [J_{k}, J^{2}] = 0,\quad \text{for } k=1,2,3
    \end{aligned}
  \end{equation}
\end{theorem}
\begin{proof}
  取$k=1$ 为例, $k=2,3$ 证明类似.
  \begin{equation}
    \begin{aligned}
      [J_{1}, J^{2}] &= J_1^{3} + J_1 J_2^{2} + J_1 J_3^{2} - J_1^{3} - J_2^{2} J_1 - J_3^{2}J_1\\
        &= {\rm i}(J_2 J_3 + J_3 J_2 - J_2 J_3 - J_3 J_2) \\
        &=0
    \end{aligned}
  \end{equation}
\end{proof}

\begin{remark}
  不难看出$J^{2}$与李代数中任意元素都对易, 这样的
  算符我们称为\emph{Casimir算符}, 我们这里不去讨论
  更深层的数学, 仅仅不证明的提及两个性质:
  \begin{itemize}
    \item Casimir算符是唯一的(除去那些相差/相乘一个常数的).
    \item 在任何不可约表示中, Casimir算符被表示为$\lambda E$的
      形式, 其中$E$ 是单位矩阵, 而$\lambda$ 是一个\emph{由表示确定的数}.
      换言之, 任何不可约表示下的矢量都是Casimir算符的特征矢量,
      并且他们具有相同的特征值.\mn{事实上这就是Schur引理.}
  \end{itemize}
\end{remark}

接下来我们引入升降阶算符:

\begin{equation}
  \begin{aligned}
    J_{\pm} = J_1 \pm {\rm i} J_2
  \end{aligned}
\end{equation}

不难发现升降阶算符存在如下性质:

\begin{property}[Properties of raising and lowering operators]
  \begin{enumerate}
    \Item \begin{equation}
      \begin{aligned}
        [J_3, J_{+}] = J_{+}
      \end{aligned}
    \end{equation}
  \Item \begin{equation}
    \begin{aligned}
      [J_3, J_{-}] = -J_{-}
    \end{aligned}
    \end{equation}
  \Item \begin{equation}
    \begin{aligned}
      [J_{+}, J_{-}] = 2 J_3
    \end{aligned}
  \end{equation}
  \Item \begin{equation}\label{eq:j^2_expansion}
    \begin{aligned}
      J^{2} = J_3^{2} - J_3 + J_{+}J_{-} = J_3^{2} + J_3 + J_{-}J_{+}
    \end{aligned}
  \end{equation}
  \Item \begin{equation}\label{eq:j_pm_conj}
    \begin{aligned}
      J_{\pm}^{\dagger} = J_{\mp}
    \end{aligned}
  \end{equation}
  \end{enumerate}
\end{property}
\begin{proof}
  按照升降算符的定义展开计算即可.
\end{proof}

现在考虑$J_3$ 的特征矢量$J_3\ket{m}=m \ket{m}$ \sn{当然, 也是$J^{2}$ 的特征矢量},
我们可以发现$J_{+}$ 的作用是将$\ket{m}$变为具有特征值$m+1$ 的$J_3$ 的特征矢量:

\begin{equation}
  \begin{aligned}
    J_3 J_{+} \ket{m} = [J_3, J_{+}]\ket{m} + J_{+} J_3 \ket{m} = (m+1) J_{+}\ket{m}
  \end{aligned}
\end{equation}

类似的, $J_{-}$ 的作用是将$ \ket{m}$变为$J_3$ 的特征值为$m-1$ 的特征矢量.

考虑到$J_3$ 的特征矢量的数目决定了表示的维数, 因此我们要求$J_{+}$ 和
$J_{-}$ 在$\ket{m}$ 上的作用必须在某一项截断, 来保证我们得到的是
一个有限维表示.

首先对于$J_{+}$, 假设在$\ket{j}$ 处截断:

\begin{equation}
  \begin{aligned}
    J_{+}\ket{j} = 0
  \end{aligned}
\end{equation}

那么, 利用\ref{eq:j^2_expansion}, 我们不难得到:
\begin{equation}
  \begin{aligned}
    J^{2}\ket{j} = j(j+1) \ket{j}
  \end{aligned}
\end{equation}

对于$J_{-}$, 假设其在$\ket{l}$ 处截断:

\begin{equation}
  \begin{aligned}
    J_{-}\ket{l} = 0
  \end{aligned}
\end{equation}

同样的, 利用\ref{eq:j^2_expansion}可以得到:

\begin{equation}
  \begin{aligned}
    J^{2}\ket{l} = l(l-1)\ket{j}
  \end{aligned}
\end{equation}

注意, 在一个确定的不可约表示下, 任何矢量对于$J^{2}$ 都具有
相同的特征值, 而这就意味着:

\begin{equation}
  \begin{aligned}
    j(j+1) = l(l-1)
  \end{aligned}
\end{equation}

上式给出两个解: $j=l-1$ 或者$l=-j$, 但是由于$j$ 是升阶算
符的截断处, 因此是最大的特征值, 仅能取$l=-j$.

同时, 由于$\ket{l}$ 可以看作利用$J_{-}$ 从$\ket{j}$ 降阶而
来, 因此一定满足:

\begin{equation}
  \begin{aligned}
    j-l = 2j = n,\quad n\in \mathbb{N}^{+}
  \end{aligned}
\end{equation}

至此, 我们总结如下:

\begin{theorem}[I.R.R of SO(3) Lie Algerbra]
  \color{black}
  ${\rm SO}(3)$的李代数的每个不可约表示可由一个确定的的半整数本征值$j$确定, 相应表示下的正交基矢给出:
  \begin{equation}
    \begin{aligned}
      &J^{2}\ket{j\,m} = j(j+1)\ket{j\,m}\\
      &J_3\ket{j\,m} = m \ket{j\,m}\\
      &J_{\pm}\ket{j\,m} = [j(j+1)-m(m\pm 1)]^{1 / 2}\ket{j\,m\pm 1}
    \end{aligned}
  \end{equation}
\end{theorem}
\begin{proof}
  (1)式和(2)式都是我们证明过的内容, 唯一需要讨论的是(3)式. 利用\ref{eq:j_pm_conj}和\ref{eq:j^2_expansion}:
  \begin{equation}
    \begin{aligned}
      \bra{j\,m} J_{-}^{\dagger} J_{-} \ket{j\,m} = \bra{j\,m}J_{+}J_{-}\ket{j\,m} = [j(j+1)-m(m-1)]\bra{j\,m}\ket{j\,m}
    \end{aligned}
  \end{equation}
  对于$J_{+}$ 同理.
\end{proof}

接下来我们来研究${\rm SO}(3)$ 群元的表示矩阵. 首先我们将
欧拉角表示重新变回固定坐标系中:

\begin{theorem}
  对于用欧拉角表示的转动$R(\alpha, \beta, \gamma)$, 可以表示
  为固定坐标系$(1,2,3)$ 下的转动:
  \begin{equation}
    \begin{aligned}
      R(\alpha, \beta, \gamma) = R_3(\alpha)R_2(\beta)R_3(\gamma)
    \end{aligned}
  \end{equation}
\end{theorem}
\begin{proof}
  首先引入一个引理\mn{我并没有找到一个对此引理的值得复述的证明, 因此在此直接使用}:\\
  \begin{lemma}
    对于绕定轴$\vec{n}$ 的旋转$R_{\vec{n}}(\alpha)$以及
    任意旋转 $R_1$, 有:
    \begin{equation}
      \begin{aligned}
        R_{\vec{n'}}(\alpha) = R_1 R_{\vec{n}}(\alpha) R_1^{-1}
      \end{aligned}
    \end{equation}
    其中$\vec{n'} = R_1 \vec{n}$.
  \end{lemma}
  现在考虑一个欧拉角描述下的旋转$R(\alpha, \beta, \gamma)$,
  三次旋转轴分别为$3, 1',3''$, 利用引理4, 我们有:
   \begin{equation}
    \begin{aligned}
      R(\alpha, \beta, \gamma) &= R_{3''}(\gamma) R_{1'}(\beta) R_{3}(\alpha)\\
        &= R_{1'}(\beta)R_{3}(\alpha+\gamma)\\
        &= R_{3}(\alpha)R_{1}(\beta)R_{3}(\gamma)
    \end{aligned}
  \end{equation}
\end{proof}

那么任意转动$R(\alpha, \beta, \gamma)$ 在$j$ 表示下的转动矩阵元为:
\begin{equation}
  \begin{aligned}
    D^{j}(\alpha,\beta, \gamma)^{m'}_{m} &= \bra{j\,m'}e^{- {\rm i} \alpha J_3}e^{- {\rm i} \beta J_2}e^{- {\rm i} \gamma J_3} \ket{j\,m}\\
    &= e^{- {\rm i} \alpha m'} \bra{j\,m'}e^{- {\rm i} \beta J_2} \ket{j\,m} e^{- {\rm i} \gamma m}
  \end{aligned}
\end{equation}
可以看出, 转动矩阵的核心在于:
\begin{equation}
  \begin{aligned}
    d^{j}(\beta)^{m'}_{m} = \bra{j\,m'}e^{- {\rm i} \beta J_2} \ket{j\,m}
  \end{aligned}
\end{equation}
