\subsection{量子熵}

现在让我们来简单讨论一下量子体系中熵的概念. 首先, 我们希望这一量子熵刻画某种不确定性,
这种不确定性是由于\emph{混合态中各态按照概率随机出现}导致的; 并且为了和热力学熵相似,
我们希望量子熵也是广延量\sn{这里的数学处理会更加复杂, 原因在于每个量子体系拥有自己独立的Hilbert空间,
而混合体系的Hilbert空间应当是各体系空间的张量积.}, 我们先定义一个算符:
\sn{此处的负号是为了使$o$ 半正定}
\begin{equation}
  \begin{aligned}
    o = -\log \rho
  \end{aligned}
\end{equation}

首先不难看出, 这一定义满足广延量的性质. 并且, 对于纯态:

\begin{theorem}
若$\rho$ 是一个描述纯态的密度算符, 则有:  
\begin{equation}
  \begin{aligned}
    o(\rho) = 0
  \end{aligned}
\end{equation}
\end{theorem}
\begin{proof}
  由于纯态的密度算符的幂等性, 有:
  \begin{equation}
    \begin{aligned}
      \log(\rho^{2}) = \log \rho + \log \rho = \log \rho
    \end{aligned}
  \end{equation}
  即$\log \rho=0$
\end{proof}

符合我们的直观(刻画的是由混合态中态的混合引入的不确定性).

那么, 这样的一个算符, 其在相应的密度算符$\rho$ 刻画的态下的期望值为:

\begin{equation}
  \begin{aligned}
    S = -{\rm tr}\; (\rho \log \rho)
  \end{aligned}
\end{equation}

